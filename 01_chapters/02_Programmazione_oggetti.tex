\chapter{Classi e oggetti, attributi e metodi}
Gli attributi determinano lo stato degli oggetti instanziati a partire da una determinata classe, il cui valore \`e particolare ad ogni istanza. I metodi definiscono 
invece il comportamento di tale oggetto, con codice condiviso per tutte le istanze di tale classe. Le istanze di una classe sono oggetti, che contengono un riferimento
ad un oggetto. Si noti che in java gli oggetti sono accessibili solo per riferimento. 
\subsubsection{Passaggio di parametri}
I parametri formati da tipi primitivi sono passati per copia, mentre i parametri costituiti da oggetti, sono passati per copia del riferimento, pertanto gli oggetti sono
sempre passati per riferimento.
\subsection{Package e information hiding}
Attributi e metodi di una classe per cui non \`e stabilita alcuna regola di visibilit\`a sono visibili solo all'interno dello stesso package in cui \`e contenuta la 
classe. Ogni compilation unit (file .java) contiene classi appartenenti allo stesso package e contiene un'unica classe public e altre private. Il package java.lang
contiene classi di uso molto comune pertanto vengono importate in automatico. 
\subsubsection{Attributi costanti}
\`E possibile creare degli attributi costanti prefissando a essi la keyword final. Le asserzioni (la terminazione di compilazione) sono determinate da un System.exit(1),
spesso precedute da un output in console. 
\subsection{Distruttore}
In java non \`e permesso distruggere esplicitamente gli oggetti, questa operazione \`e compiuta in automatico dal garbage collector, che interviene automaticamente quando
si rende necessaria della memoria, liberando quella occupata da oggetti la cui referenza non \`e pi\`u attiva. Pu\`o essere esplicitamente attivato attraverso System.gc()
che di norma non \`e necessario. \`E possibile definire nel metodo finalize azioni da eseguire all'atto della distruzione dell'oggetto. Non \`e un distruttore \`e 
utilizzato per rilasciare risorse diverse alla distruzione di un oggetto.
\subsection{Creazione di oggetti}
I nuovi oggetti sono creati con l'operatore new, che richiede la chiamata di un particolare metodo: il costruttore che alloca la memoria necessaria all'oggetto e 
inizializza lo spazio allocato. Il costruttore ha lo stesso nome della classe, non indica il tipo di ritorno e non ha return, \`e possibile fare dell'overloading. Di 
default viene creato un costruttore che si limita ad allocare lo spazio necessario alla memoria. 
\subsubsection{Tipi array}
Sono anch'essi tipi riferimento come le classi. Dato un tipo T, un array viene diciarato con T[]. In mancanza di inizializzazione la dichiarazione di un'array non 
alloca spazio per i suoi elementi. L'allocazione si crea dinamicamente attraverso new T[dimensione]. Se non sono di tipo primitivo, alloca spazio per i riferimenti. 
System.arraycopy(src, int srcPos, dst, int dstPos, int length) copia length elementi da src a partire da srcPos in dst a partire da dstPos.