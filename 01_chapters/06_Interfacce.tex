\chapter{Interfacce}
Un'interfaccia \`e una classe composta interamente da metodi astratti, senza attributi. Un'interfaccia pu\`o contenere costanti, spesso si usano interfacce completamente
vuote per taggare classi con speciali propriet\`a. Un'interfaccia pu\`o ereditare da una o pi\`u interfacce. Una classe pu\`o implementare una o pi\`u interfacce e 
implementarne tutti i metodi astratti a meno che sia astratta. L'interfaccia si limita a definire il comportamento che una classe deve avere. Si utilizza per gestire il
problema delle multiple inheritance in modo da non rischiare di incorrere in metodi con la stessa firma. Un'interfaccia pu\`o essere utilizzata per definire il tipo 
di una variabile ma non per creare un oggetto.
\section{Collections}
Una collection \`e un oggetto che raggruppa elementi multipli in una singola entit\`a. A differenza degli array gli elementi possono essere eterogenei e la sua lunghezza
non \`e fissata. Sono utilizzate per immagazzinare, recuperare e traseferire gruppi di dati da un metodo all'altro. Il java collection framework contiene tre tipi di elementi: interfacce per specificare diversi tipi di servizi associati a diversi tipi di collection, potenzialmente associate a diverse strutture dati, implementazioni 
di specifiche strutture dati di uso comune che implementano le interfacce e algoritmi, codificati in metodi, che implementano operazioni comuni a pi\`u strutture dati. 
Come per esempio ricerca ordinamento, mescolamento, composizione. Lo stesso metodo pu\`o essere utilizzato in diverse implementazioni. 
\subsection{Interfacce}
\begin{enumerate}
\item Collection, un'arbitraria collezione di elementi.
\begin{itemize}
\item Set, collection senza duplicati.
\begin{itemize}
\item SortedSet, set su cui \`e definito un ordinamento.
\end{itemize}
\item List, collection in sequenza ordinata, accessibili tramite indice, sono ammessi duplicati.
\item Queue, collection in coda in cui sono ammessi duplicati.
\begin{itemize}
\item Deque queue a cui \`e possibile accedere da entrambe le estremit\`a.
\end{itemize}
\end{itemize}
\item Map, una collezione di coppie chiave-valore senza duplicati.
\begin{itemize}
\item SortedMap, map su cui \`e definito un ordinamento sulle chiavi. 
\end{itemize}
\end{enumerate}
\subsection{Operazioni base}
\begin{itemize}
\item int size();
\item boolean isEmpty();
\item boolean contains(Object element); 
\item boolean add(Object element); (true se la collection \`e cambiata)
\item boolean remove(Object element); (true se la collection \`e cambiata)
\item Iterator iterator();
\end{itemize}
\subsection{Bulk operation}
Consentono di effettuare operazioni su pi\`u elementi contemporaneamente.
\begin{itemize}
\item boolean containsAll(Collection c); 
\item boolean addAll(Collection c); (true se la collection \`e cambiata)
\item boolean removeAll(Collection c); (true se la collection \`e cambiata)
\item boolean retainAll(Collection c);
\item void clear();
\item Object[] toArray();
\end{itemize}
