\chapter{Polimorfismo}
Con polimorfismo si intende la capacit\`a di un elemento sintattico a riferirsi a elementi di diverso tipo. 
\subsection{Principio di sostituzione di Liskov}
Se S \`e un sottotipo di T, allora le variabili di tipo T in un programma possono essere sostituite con variabili di tipo S senza alterare alcuna propriet\`a desiderabile
del programma. Pertanto una variabile di tipo T pu\`o riferirsi a oggetti di tipo T o di un suo sottotipo, allo stesso modo per i parametri formali di una funzione. 
In java il legame tra un oggetto e il suo tipo \`e detto dynamic binding (o lazy/late binding). In presenza di polimorfismo si distingue tra tipo statico, dichiarato a 
compile time e tipo dinamico, assunto a runtime. In java la chiamata di una funzione dipende dal tipo dinamico di un oggetto e non dal suo tipo statico. Il compilatore
determina la firma del metodo da eseguire basandosi unicamente sul metodo statico. In caso di overriding la specifica implementazione del metodo la cui firma \`e stata 
decisa dal compilatore viene scelta basandosi sul metodo dinamico. 
\subsection{Regole per il binding}
Chiamato un metodo che dipenda dal tipo dell'oggetto si cerca all'interno della classe del tipo statico dell'oggetto il metodo con la firma pi\`u vicina all'invocazione.
Successivamente si guarda al tipo dinamico dell'oggetto e si nota se \`e stato fatto l'override di tale metodo scelto. Se s\`i si usa l'implementazione del tipo dinamico.
In Java le decisioni sono sempre a runtime salvo quando sia possibile decidere univocamente a compile time per metodi private, static e final e costruttori. 