\chapter{Implementation of a language}
Suppose to have an abstract machine $\mathcal{M_L}$ that understands the machine language $L$. Every abstract machine can understand only one language, but a language can be understood
by different machines. To implement the language $\mathcal{L}$ means to realize an abstract machine $\mathcal{M_L}$ that can execute programs written in this language. The implementation
can be done via hardware, software or firmware.
\section{Implementation in software}
There are two implement $\mathcal{M_L}$ in software, meaning you need it to run on a host machine ($\mathcal{MO_{LO}}$): interpreted or compiled.
\subsection{Interpreted}
By this approach there is a program written in $\mathcal{LO}$ that understands and executes $\mathcal{L}$ and implements the cycle fetch, decode, load, execute and save. This program
called the interpreter translates from $\mathcal{LO}$ to $\mathcal{L}$ instruction by instruction.
\subsection{Compiled}
By this approach there is a program that can translate other programs from $\mathcal{L}$ to $\mathcal{LO}$. This program is called the compiler which translate the entire program before
execution and can be executed from a different machine.
\subsection{Hybrid implementation}
By this approach there is a compiler that translates the program into an intermediate language ($\mathcal{LI}$). This new program is then interpreted by a $\mathcal{MO_{LO}}$ program
written in $\mathcal{LI}$. Depending on the differences between $\mathcal{LI}$ and $\mathcal{LO}$ it is said that an implementation is mainly compilative (like C) or mainly interpretive
(like Java).
\subsection{Comparison between the approaches}
The purely interpretive approach has a less efficient implementation of $\mathcal{M_L}$ but it is more flexible and portable and debugging is simpler with interpretation at run-time.
The purely compilative has a more efficient implementation of $\mathcal{M_L}$ but more complex due to the difference between the languages and debugging is usually harder.
\section{Implementation in firmware}
It is the intermediate implementation between hardware and software, in which the abstract machine is inplemented as a microinterpreter and the program cycle is implemented using 
microinstruction invisible to normal users. The data structures and algorithms are realized as microprograms. This is more flexible than a pure hardware implementation.
\section{Formal definition}
A program can be seen as a function written in the language $\mathcal{L}$: $P^\mathcal{L}:\mathcal{D}\rightarrow\mathcal{D}$. If the program is $P^\mathcal{L}(i)=o$, $i\in\mathcal{D}$ is
the input, $o\in\mathcal{D}$ is the output. Interpreters and compilers are program whose input and ouput are other programs.
\subsection{Definition of an interpreter}
An interpreter for a language $\mathcal{L}$ written in language $\mathcal{LO}$ implements an abstract machine $\mathcal{M_L}$ and is the function:
\begin{equation}
\mathcal{I_L^{LO}}:(\mathcal{PR^L}\times\mathcal{D})\rightarrow\mathcal{D}
\end{equation}
Where $\mathcal{PR^L}$ is the set of programs $\mathcal{P^L}$ written in the language $\mathcal{L}$. So $\forall i\in\mathcal{D},\mathcal{I_L^{LO}}(\mathcal{P^L},i)=\mathcal{P^L}(i)$ 
meaning that for any input the interpreter applied to the program and the input returns the same result as the program applied to the input. If $\mathcal{P^L}(i)$ is calculated by 
the abstract machine $\mathcal{M_L}$, $\mathcal{I_L^{LO}}(\mathcal{P^L},i)$ is calculated by $\mathcal{MO_{LO}}$.
\subsection{Definition of compiler}
A compiler from the language $\mathcal{L}$ to the language $\mathcal{LO}$ written in language $\mathcal{LA}$, which implements $\mathcal{M_L}$ on $\mathcal{MO_{LO}}$ transforms a program
$\mathcal{P^L}\in \mathcal{PR^L}$ into a program $\mathcal{P^{L=}}\in \mathcal{PR^{LO}}$. It is described by the function:
\begin{equation}
\mathcal{C_{L,LO}^{LA}}: \mathcal{PR^L}\rightarrow \mathcal{PR^{LO}}
\end{equation}
Meaning that $\forall i\in D: \mathcal{P_C^{LO}}=\mathcal{C_{L,LO}^{LA}}(\mathcal{P^L})\Rightarrow \mathcal{P_C^{LO}}(i)=\mathcal{P^L}(i)$, that describes the fact that for any input $i$,
the compiled program applied to $i$ returns the same result as the non compiled program applied to $i$. $\mathcal{P^L}(i)$ is calculated by $\mathcal{M_L}$, $\mathcal{P^{LO}_C}(i)$ is calculated by $\mathcal{MO_{LO}}$ and the compiled program $\mathcal{C^{LA}_{L,LO}}(\mathcal{P^L})$ is calculated by $\mathcal{MA}$