\chapter{Astrazione dei controlli}
\section{Funzione di ordine superiore}
Dei linguaggi permettono il passaggio di funzioni come parametro o il ritorno di funzioni per un'altra funzione. Per gestire l'ambiente di questi casi si pu\`o usare un 
puntatore al record di attivazione nello stack nel caso di funzioni come parametri. Nel caso di funzioni come risultato si deve mantenere un record di attivazione della funzione
esterno allo stack. 
\subsection{Regole di binding}
Quando una funzione \`e passata come parametro questo crea una reference tra il nome del parametro formale e il parametro attuale. L'ambiente non locale utlilzzato dipende 
dal tipo di binding utilizzato.
\subsubsection{Deep binding}
In questo caso viene utilizzato l'ambiente esistente al momento della creazione del link, utilizzato sempre in caso di scoping statico.
\subsubsection{Shallow binding}
In questo caso viene l'ambiente esistente al momento della chiamata, pu\`o essere utilizzato con lo scoping dinamico. 
\subsubsection{Chiusura}
Sia il link al codice della funzione che del suo ambiente sono passati come parametro, la procedura passata come parametro alloca il suo record di attivazione e considera
il puntatore sulla catena statica della chiusura. In caso di utiilizzo di chiusura si perdono le propriet\`a LIFO dello stack, pertanto non esiste deallocazione automatica, 
i record di attivazione vengono salvati nello heap con catene statiche o dinamiche che li connettono e quando necessario viene chiamato un garbage collector. 
\section{Eccezioni}
Vengono utilizzate quando si riscontra un problema che rende impossibile continuare la computazione, si esce pertanto dal blocco, si passano i dati con un jump e si ritorna
il controllo al punto di attivazione pi\`u recente. I record di attivazione non pi\`u necessari sono deallocati e le risorse non pi\`u necessarie sono liberate. Le eccezioni 
si basano su due costrutti: la dichiarazione del gestore delle eccezionie il comando che genera l'eccezione. Se l'eccezione non \`e gestita dalla funzione corrente questa 
termina e l'eccezione \`e generata ancora alla funzione chiamante, alla fine si trova il gestore dell'eccezione o uno di default. I frame corrispondenti sono rimossi dallo stack 
e i valori sono ripristinati. Ogni routine contiene un handler nascosto che ripristina i valori e propaga l'eccezione. 
\subsection{Implementazione delle eccezioni}
All'inizio di ogni blocco protetto viene inserita sullo stack del chiamante, quando viene generata si rimuove il chiamante dallo stack e si controlla se \`e giusto, altrimenti
si ripete. \`E inefficiente se non si trova un handler, una soluzione migliore \`e la tavola degli indirizzi. 
