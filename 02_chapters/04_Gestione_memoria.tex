\chapter{Gestione della memoria}
Ci sono tre tipi principali per gestire la memoria: 
\begin{itemize}
\item Statico, in cui la memoria viene allocata durante la compilazione.
\item Dinamico, in cui la memoria viene allocata a runtime, si divide in:
\begin{itemize}
\item Stack, in cui gli oggetti sono allocati secondo una FIFO.
\item Heap, in cui ogni oggetto pu\`o essere allocato e disallocato ad ogni momento.
\end{itemize}
\end{itemize}
\section{Allocazione statica}
Utilizzando l'allocazione statica un oggetto ha un indirizzo assoluto che mantiene per tutta la durata dell'esecuzione.Viene solitamente utilizzata per salvare variabili \\
globali, variabili locali in assenza di ricorsione, costanti determinate a runtime e tabelle per garbage collection e type-checking. Presenta un problema per la ricorsione, 
in quanto salvando sempre nello stesso indirizzo i parametri attuali per una funzione questi vengono persi dopo la chiamata ricorsiva.
\section{Allocazione dinamica}
\subsection{Stack}
Per ogni istanza a runtime di un sottoprogramma, come in ogni blocco,  esiste un record di attivazione (frame) che contiene le informazioni dell'istanza. La struttura dati
naturale per gestire i record \`e una coda LIFO, uno stack pu\`o essere utilizzato anche in linguaggi senza ricorsione per ridurre il consumo di memoria.
\subsubsection{Record di attivazione}
Il record di attivazione consiste di tre parti:
\begin{enumerate}
\item Puntatori alla catena dinamica.
\item Variabili locali.
\item Risultati intermedi.
\end{enumerate}
L'implementazione dell'allocazione dinamica di uno stack utilizza una sequenza di chiamate, un prologo, un epilogo e una sequenza di ritorno. L'indirizzo del record di 
attivazione non \`e conosciuto a compile time, e il suo puntatore punta al record di attivazione del blocco attivo. Le informazioni nel record di attivazione sono accessibili
grazie ad un offset determinato staticamente, grazie alla somma con l'indirizzo del record di attivazione, operazione compiuta attraverso una sola operazione della macchina.
\subsubsection{Push e Pop}
All'entrata in un blocco, viene creato un nuovo linck dinamico (puntatore  al record precedente sullo stack) dal nuovo blocco al blocco precedente, il record di attivazione 
viene settato a puntare al nuovo record. All'uscita dal blocco viene elimitato il puntatore al record e settato un nuovo record di attivazione alla cima dello stack.
\subsubsection{Necessit\`a dello stack}
Per la gestione di blocchi annidati non \`e uno stack, il compilatore pu\`o analizzare tutti i blocchi e allocare memoria per tutte le variabili, causando un maggior consumo di 
memoria, ma rimanendo pi\`u efficiente non perdendo risorse per la gestione dello stack. 
\subsection{Heap}
Lo heap \`e una regione di memoria in cui blocchi e sottoblocchi possono essere allocati e disallocati in momenti arbitrari. Questo processo si rende necessario quando si trova
un'allocazione di memoria esplicita a runtime, ci sono ogetti di dimensione variabile e il loro tempo di vita non \`e LIFO. La gestione dello heap presenta dei problemi: si 
necessita di allocare efficientemente lo spazio (bisogna evitare la frammentazione) e tenere conto della velocit\`a di accesso. 
\subsubsection{Blocchi di grandezza fissa}
In questo caso lo heap \`e diviso in blocchi di grandezza fissa all'inizio linkati in una free list. Durante l'allocazione uno o pi\`u blocchi vengono eliminati dalla free list
e nella deallocazione vi sono reinseriti. 
\subsubsection{Blocchi a grandezza variabile}
Lo heap contiene inizialmente un blocco singolo. Durante l'allocazione viene ricercato un blocco della dimensione appropriata che viene poi ripristinato al blocco libero 
durante la deallocazione. 
\subsubsection{Gestione dello heap}
\textbf{Frammentazione interna}: se si necessita di allocare uno spazio di dimensione $X$ ma la dimensione del blocco \`e $Y>X$, viene sprecato $Y-X$ spazio.\\
\textbf{Frammentazione esterna}: lo spazio necessario \`e disponibile ma inaccessibile in quanto diviso in blocchi di dimensione troppo piccola. 
\subsubsection{Gestione della free list}
\textbf{Lista singola}: all'inizio esiste un blocco che occupa tutto lo heap ed ad ogni richiesta di allocazione viene ricercato un blocco della dimensione appropriata secondo
uno dei due possibili paradigmi:
\begin{itemize}
\item Best fit: ricercato il blocco pi\`u piccolo che pu\`o contenere l'oggetto. Buon utilizzo della memoria ma pi\`u lento.
\item First fit: viene ricercato il primo blocco libero. Veloce con pessimo utilizzo della memoria.
\end{itemize}
Se il blocco \`e molto pi\`u largo di quanto si necessita \`e diviso in due pezzi e il blocco inutilizzato \`e riagiunto alla free list. L'allocazione \`e lineare nel numero
dei blocchi liberi. \\
\textbf{Liste multiple}: che possono essere implementate staticamente; dinamicamente in due modi:
\begin{itemize}
\item Buddy system: con $k$ liste con la $k$-esima lista con blocchi di dimensione $2^k$:
\begin{itemize}
\item Se la dimensione di $2^k$ \`e richiesta ma non \`e disponibile si alloca un blocco di dimensione $2^{k+1}$ diviso in due.
\item Quando un blocco di dimensione $2^k$ viene deallocato, se esiste un'altra met\`a libera viengono uniti. 
\end{itemize}
\item Sistema di Fibonacci: analogo al Buddy System ma con l'utilizzo della sequenza di Fibonacci. 
\end{itemize}
\section{Implementazione delle regole di scoping}
\subsection{Scoping statico}
\subsubsection{Catena statica}
Viene determinato il primo activation record in cui la variabile si trova, l'accesso viene eseguito tramite l'offset con questo record di attivazione. Questo pu\`o essere
determinato dal link dinamico visto sopra o attraverso un link statico al blocco che contiene il testo del blocco corrente. Se il link dinamico dipende dalla sequenza di 
esecuzione, il link statico dipende dall'annidamento statico della dichiarazione della funzione. Il link statico \`e determinato dal chiamante che possiede le informazioni
riguardante l'annidamento statico determinato dal compilatore e il proprio record di attivazione. Se il chiamato \`e automaticamente incluso nel chiamante, questo passa il 
proprio static pointer al chiamato, altrimenti il puntatore statico viene trovato dopo $k$ passaggi nella catena statica.
\subsubsection{Display}
Viene utilizzato per ridurre il costo di far scorrere la catena ad una costante: \`e un array contenente la catena in cui l'i-esimo elemento \`e un puntatore al record di
attivazione al sottoprogramma a livello di annidamento $i$. Se il sottoprogramma si trova a livello $i$ e l'oggetto in uno scope esterno a livello $h$, questo pu\`o essere
trovato nel display al livello $i-h$. Quando il chiamante chiama una funzione a livello di nesting $i$ salva il valore di $display[i]$ nel proprio record di attivazione e setta
il puntatore ad una copia del suo nuovo record di attivazione. Implementazioni moderne raramente utilizzano questo metodo in quanto non si trovano spesso catene statiche pi\`u 
lunghe di $3$. 
\subsection{Scoping dinamico}
L'associazione tra nomi e oggetti dipende dal flusso di controllo a runtime e dall'ordine in cui i sottoprogrammi sono chiamati. Di base l'associazione \`e determinata 
dall'ultima associazione creata che non \`e stata ancora distrutta. I nomi vengono salvati nel record di attivazione e i nomi vengono chiamati attraverso uno stack. 
\subsubsection{A-list}
Le associazioni sono salvate in una struttura apposita gestita come uno stack. \`E semplice da implementare, tutti i nomi sono listati esplicitamente. I costi di gestione sono
rappresentati dall'accesso uscita di un blocco e il conseguimento inserimento o rimozione da uno stack. Il costo di accesso \`e linare con la profondit\`a della lista.
\subsubsection{Central table of reference (CRT)}
La tavola salva tutti i nomi distinti del programma staticamente con accesso in tempo costante o grazie all'utilizzo di funzioni di hash. Ogni nome ha una lista di 
associazioni: con come primo il pi\`u recente, seguito dagli inattivi. \`E pi\`u complesso di una A-list ma utilizza meno memoria in quanto tutti i nomi sono salvati un'unica
volta e se sono utilizzati staticamente non sono necessari, il costo di gestione \`e rappresentato dall'entrata e uscita di un blocco che rende necessario la gestione della 
lista di tutti i nomi presenti nel blocco. Il tempo di accesso \`e costante. 